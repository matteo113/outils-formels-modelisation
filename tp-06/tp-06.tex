\documentclass[a4paper, titlepage]{article}
\usepackage[round, sort, numbers]{natbib}
\usepackage[utf8]{inputenc}
\usepackage{amsfonts, amsmath, amssymb, amsthm}
\usepackage{color}
\usepackage{framed}
\usepackage{listings}
\usepackage{mathtools}
\usepackage{paralist}
\usepackage{parskip}
\usepackage{subfig}
\usepackage{tikz}
\usepackage{titlesec}
%\usepackage{ulem}

\numberwithin{figure}{section}
\numberwithin{table}{section}

\usetikzlibrary{arrows, automata, backgrounds, petri, positioning}
\tikzstyle{place}=[circle, draw=blue!50, fill=blue!20, thick]
\tikzstyle{transition}=[rectangle, draw=black!50, fill=black!20, thick, minimum width=5.5mm, minimum height=5.5mm]

% define new commands for sets and tuple
\newcommand{\setof}[1]{\ensuremath{\left \{ #1 \right \}}}
\newcommand{\tuple}[1]{\ensuremath{\left \langle #1 \right \rangle }}
\newcommand{\card}[1]{\ensuremath{\left \vert #1 \right \vert }}

\definecolor{lstbg}{rgb}{1,1,0.9}
\lstset{basicstyle=\ttfamily, numberstyle=\tiny, breaklines=true, backgroundcolor=\color{lstbg}, frame=single}
\lstset{language=C}

\makeatletter
\newcommand\deadline[1]{\def\@deadline{#1}}
\newcommand\objective[1]{\def\@objective{#1}}
\newcommand{\makecustomtitle}{%
	\begin{center}
		\huge\@title \\
		[1ex]\small Dimitri Racordon, le \@date
	\end{center}
	\begin{framed}\@deadline\end{framed}
	\begin{framed}\@objective\end{framed}
}
\makeatother

\begin{document}

\title{Outils formels de Modélisation \\ 6\textsuperscript{ème} Travail personnel}
\author{Dimitri Racordon}
\date{12.12.17}

\deadline{
\textbf{Date de rendu}: Jeudi 11.01.18 à 23h55 \\

  Comme leur nom l'indique, ces travaux sont \emph{personnels}.
  La copie est strictement interdite, et toutes similitudes entre deux rendus
  seront santionnées par la note de 0.
  Tout dépassement de la date et heure de rendu sera lourdement pénalisée.
  Date et heure de rendu sont toujours données en heure locale de Genève.
  Tout commit sur votre dépôt publié après la date de rendu sera ignoré.

  Seule la branche \texttt{master} de votre dépôt sera prise en compte
  lors de la correction.

  Votre travail devra être rendu sous la forme d'un fichier \textbf{pdf},
  déposé à la racine du répertoire \texttt{tp-06}.
}

\objective{
  Dans ce TP, vous allez étudier la logique du premier ordre
  en vous penchant sur un problème de relations dans un graphe.
}

\makecustomtitle

\section{Dans la peau d’un apollon}

Forkez le dépôt https://github.com/cui-unige/outils-formels-modelisation.
Vous travaillerez sur votre propre version du dépôt,
et effectuerez tous vos commit sur ce dépôt-ci.

Le projet pour ce TP se trouve dans le répertoire \texttt{tp-06}.

\section{Dans la peau d'un apollon}

Depuis quelques jours,
un ami ne cesse de vous parler d’une nouvelle série qu’il souhaite absolument vous faire regarder:
\emph{Dans la peau d’un apollon}.
A force d’insistance, vous avez fini par céder et l’avez ajoutée à votre liste Netflix.

Synopsis:
\emph{
  Dans la peau d’un apollon est une série palpitante qui suit l’histoire d’Alexandre,
  un jeune homme au charisme sans égal.
  Alexandre (ou Alex pour les intimes) est en couple depuis plusieurs années avec Alexandrine
  (ou Alex pour les intimes).
  Malheureusement, sa relation n’est pas toujours au beau fixe.
  C’est en tout cas ce qu’il confie à ses deux meilleurs amis, Robin et Miguel.
  Robin quant a lui connait aussi quelques tumultes dans sa relation avec Floriane.
}

Ce soir, vous aviez prévu une sortie avec quelques amis,
mais la soirée semble être tombée à l’eau à cause de divers empêchements.
Seul(e) chez vous et n’ayant rien d’autre de prévu,
vous décidez donc de regarder quelques épisodes ce cette fameuse série.
Avant de démarrer, vous vous remémorez ce que votre ami vous en avait dit:

\begin{itemize}
  \item Alex est en couple avec Alex et Robin est en couple avec Floriane.
  \item Il y a une femme et un homme qui aiment leur partenaire respectif mais qui ont aussi des sentiments pour une autre personne.
  \item Il y a une femme et un homme qui n’aiment que leur partenaire respectif.
  \item Après une soirée de folie dans l’épisode 4, Miguel commence à éprouver des sentiments pour une personne qui aime une personne qui aime Alexandrine.
  \item C’est un peu sexiste parce que toutes les femmes n’aiment que des hommes.
  \item Robin aime une personne dans un triangle amoureux.
  \item Personne ne s'aime soi-même.
\end{itemize}

Traduisez les informations ci-dessus en formules de logique du premier ordre,
puis proposez une solution de graphe de relations amoureuse qui satisfait toutes ces contraintes.

\section{Saison 2}

Vous aviez tort d’être sceptique.
\emph{Dans la peau d’un apollon} est une série incroyable
et vous êtes devenu(e) fan absolu(e) des personnages.
En attendant avec impatience la sortie de la saison 2,
vous vous êtes lancé(e) dans une discussion passionnée avec votre ami
pour tenter de prédire les twists des épisodes à venir.

Selon les dernière rumeurs, il paraîtrait qu’un des hommes serait en fait le frère caché d’Alex!
Votre ami vous soutient que les producteurs n’iraient pas jusqu’à mettre en scène une relation incestueuse,
mais est également convaincu que Miguel est secrètement amoureux de Floriane.
Prouvez lui à l’aide des séquents qu’il a forcément tort:
soit il y a une relation incestueuse, soit Miguel n’est pas amoureux de Floriane.

\end{document}
